\documentclass[fontsize=5pt]{scrartcl}

%
% Original Page by LinuxMercedes
%

\usepackage[
        nohead,
        nofoot,
        left=0.55in,
        right=0.55in,
        top=0.55in,
        bottom=0.55in,
]{geometry}

\usepackage{amsmath,scalefnt}

\renewcommand*{\arraystretch}{.5}

\usepackage{multicol}
\setlength{\columnsep}{5pt}

\usepackage{helvet}
\renewcommand{\familydefault}{\sfdefault}

\pagenumbering{gobble}

\usepackage{enumitem}
\setlist[itemize]{itemsep=-2pt, itemindent=0pt, leftmargin=*}
\setlist[enumerate]{itemsep=-2pt, itemindent=0pt, leftmargin=*}

\usepackage[compact]{titlesec}
\titlespacing{\section}{-1pt}{-1pt}{-1pt}
\titlespacing{\subsection}{-1pt}{-1pt}{-1pt}

\usepackage{listings}

%Y hoy yo reza que no empleo ni alma pobre encontrará esta magica negra.
%This is a custom 'tight' matrix for this cheatsheet. It's ugly.
\newenvironment{tmatrix}%
{ 
  %\scalefont{.5}
  %\setlength{\tabcolsep}{5pt}
  $\left[\hspace{-3.5pt}\begin{array}{c@{\hspace{1pt}}c@{\hspace{1pt}}c@{\hspace{1pt}}|@{\hspace{0pt}}c}
}%
{
   \end{array}\hspace{-3.5pt}\right]$
}

\newenvironment{tmatrix3}%
{ 
  %\scalefont{.5}
  %\setlength{\tabcolsep}{5pt}
  $\left[\hspace{-3.5pt}\begin{array}{c@{\hspace{1pt}}@{\hspace{1pt}}c@{\hspace{1pt}}c@{\hspace{3pt}}}
}%
{
   \end{array}\hspace{-3.5pt}\right]$
}

\newenvironment{tmatrix1}%
{ 
  $\left[\hspace{-3.5pt}\begin{array}{c@{\hspace{3pt}}}
}%
{
   \end{array}\hspace{-3.5pt}\right]$
}


%was 3 3 3 3
\DeclareMathSizes{3pt}{3pt}{3pt}{3pt}

\begin{document}

\begin{multicols}{3}
  \section{Definitions}
    \begin{itemize}
      \item \textbf{Operating Systems}: exploit hardware resources (many processors), provide a set of services to users, manage secondary memory and I/O, %
          and provide networking/comm. support. Elements of an OS:
      \begin{itemize}
        \item Processor
        \item Main Memory (real/primary)
        \item I/O Modules: secondary memory devices, comms equipment, terminals
        \item System bus: communication among processors, memory, and I/O modules
       \end{itemize}
       \item \textbf{Interrupts}: interrupts the normal sequence of execution. Improve processing efficiency and lets the processor not wait on things.
       \item \textbf{ISR}: Interrupt service routine, handles interrupts. Whenever  there is an interrupt, control is transferred to this program.  %
                       It determines the nature of the interrupt and performs the necessary actions to handle it.

    \end{itemize}
     \subsection{Processing}
      \begin{itemize}       
       \item \textbf{Multithreading}: Executing an application using parts of it, which were divided into threads. 
       \item \textbf{Thread}: Dispatchable unit of work, has its own data area and processor context, execute sequentially.
       \item \textbf{Process}: A collection of one or more threads and resources.
       \item \textbf{SMP}: Symmetrix MultiProcessing, multiple, connected, homogenus processors using shared memory.
                           Threads can be scheduled across all processors. Also refers to the hardware model it runs on, with
                           many processors.
        \item \textbf{Deadlocks}: One process is waiting on another to finish.
        \item \textbf{Starvation}: A process is postponed indefinitely and does not wake.
      \end{itemize}    

  \section{Diagrams}
    \begin{enumerate}
      \item Process Interrupt Cycle: 
      \item semaphore example from 5 slide 12
    \end{enumerate}
    
   \section{Interrupts}
      \begin{itemize}
        \item \textbf{Interrupt Types}: Program interrupts (arithmetic overflow, division by zero, execute illegal instruction), timer, I/O interrupt, HW failure.
      \end{itemize}
    \section{Unix}
      \begin{itemize}
       \item Processor States:
        \begin{itemize}
         \item User Running: executing in user mode.
         \item Kernel Running: executing in kernel mode.
         \item Ready to run, in memory: Ready to run as soon as the kernel schedules it.
         \item Asleep in memory: Unable to execute until an event occurs; process is in main memory, blocked state.
         \item Ready to run, swapped: Ready to run but moved off main memory.
         \item Sleeping, swapped: awaiting an event but moved off main memory.
         \item Preempted: process is returning from kernel to user mode, but the kernel preempts it and does a process switch
               to schedule another process. Very similar/same as ready to run, in memory.
         \item Created: newly created and ready to run.
         \item Zombie: no longer exists and left a record for its parent to clean up.
        \end{itemize}
        \item Unix Modes: user mode, less privileged, usually user programs. System/supervisor/kernel: privileged.
      \end{itemize}
      
     \section{Threads}
      \begin{itemize}
       \item Many approaches: 
       \begin{itemize}
        \item Single Thread (DOS). One process, one thread.
        \item Multithreaded: Java ``green'' threads, one process many threads.
        \item Unix variants: many processes one thread each.
        \item Windows/Solaris/Unix: Many processes, many threads each.
       \end{itemize}
       \item Threads have their own execution state (running/ready), state is saved when not running, execution stacks, 
             some of their own storage for variables, and access to all memory and resources of its process.
       \item Way cheaper to make and destroy than a new process due to shared memory.
       \item Switching to a thread is cheaper.
       \item {{POSIX THREAD PLACEHOLDER}}
      \end{itemize}
      
      \section{Concurrency}
        \begin{itemize}
          \item Mutual Exclusion: Only one process can be in a critical section at a time. Without this, results aren't consistent.
          \item Critical Section: CS. Used a lot to encapsulate important parts of code.
  
         \begin{itemize}
          \item Disabling interrupts: this guarantees mutual exclusion but is exploitable for user processes. 
          \item Lock variables: Processes must have a lock to run or modify variables (GIT). Sleep in process while locked
                get(lock), run our code, and release(lock).
          \item Semaphores, almost universally using Dijkstra's Semaphore.
         \end{itemize}

       \end{itemize}
       \subsection{Semaphores}
         An unsigned number variable, s, which can only be changed or tested by two atomic functions:
         \begin{itemize}
          \item $P(s)$: [ while(s==0) {wait}; s = s-1; ]
          \item $V(s)$: [ s = s + 1; ]
         \end{itemize}
         Semaphores are:
         \begin{itemize}
          \item Conceptualized and have no specific purpose in mind. However, for us, P is get and
                V release.
          \item Semaphore locks will add items to a queue as they wait for the lock to release. See Semaphore example.
          \item Starvation and Deadlocks do not happen in Semaphore locks when used properly
          \item However if two locks in a semaphore are waiting on each other, this deadlocks.
        \end{itemize}
       \subsection{Producer/Consumer Problem}
         


      
  \section{Lectures}
    
      
  \section{Glossary}
    \subsection{Something here}
    
  \section{QA Pool}
    \subsection{T/F - Process Dynamics}
        \begin{itemize}
          \item Multiprogramming (having more programs in RAM simultaneously) decreases total CPU efficiency: false. Jobs can be chosen as things have to wait, having several
                in memory increases your capability of this. CPU will never be idle.
          \item Unlike \textbf{time sharing systems}, the principal objective of \textbf{batch processing systems} is to minimize response time. False: principal objective for time sharing is to 
                minimize response time. Whereas, the principal objective for batch processing is to maximize processor use.
          \item A forked process shares its parent's data space at all times: false.
          \item A forked process may share code space with its parent. true.
          \item A forked process may share a Run-Time Stack with its parent. false.
          \item Context switch means a process switching from a ``blocked state'' to ``ready state''. false
          \item In the ``zombie'' state, the process no longer exists but it leaves a record for its parent process to collect. true.
          \item While DMA (Direct Memory Access) is taking place, processor is free
                to do other things. The processor is only involved at the beginning
                and end of the DMA transfer. True
          \item If a process uses up its allocated time slot, a timer interrupt occurs
                and the process is placed in a BLOCKED Queue. False, ready.
          
                
          
        \end{itemize}
    
    \subsection{OS Related Questions}
      \begin{itemize}
          \item Many modern O.S. use a \textbf{microkernel} design, what does that mean? Microkernel is a small privileged OS core that provides process
                scheduling, memory management, and communication services and relies on other processes to perform some of the functions 
                traditionally associated with the operating system kernel. It's slimmer.
          \item What are \textbf{monolithic} kernels? Old OSs had Monolithic kernels which are large kernels containing
                virtually the complete operating system, including scheduling, file 
                system, device drivers, and memory management.  All the functional
                components of the kernel have access to all of its internal data
                structures and routines. Examples of such systems are: UNIX, MS-DOS.
          \item Linux is a monolithic kernel: true but it allows user-installable modules.
          \item What is dual mode operation? User mode vs supervisor mode.
          \item Why does a machine need dual mode operation? To ensure proper operation, we must protect the operating system and all 
                the programs and their data from any malfunctioning program. 
                The protection is accomplished by designating some of the machine
                instructions that may cause harm as "privileged" instructions.
                Dual mode operations can protect the OS from errant users, and the users from one another. 
                It also prevents the abuse of priviledged instructions (such as 'interrupt enable/disable')
                by the user programs.
          \item What is a \textbf{context switch}? Switching the CPU to another process.
          \item What information is saved and restored during a context switch? Context switch requires saving the state of the old process and
                loading the saved state for the new process. Process state
                minimally includes current contents of registers, program counter,
                stack pointer, file descriptors, etc. 

          \item Why are context switches considered undesirable (to be minimized) by OS designers? They waste a great deal of CPU time when save/load states.
          \item \textunderscore{Unix process tree here}
          \item Give an example event for each transition to occur.(You need to
                give at least five example events).
                %{create process}: user starts a new process or a program
                %  issues a fork() or a pthread_create() call.
                %{cpu available}: the process currently occupying CPU stops running
                %  and the process in front of the READY queue is scheduled to run
                %{time expires}: currently running process uses up its alloted time
                %  slot for this turn (timer interrupt).
                %{event wait}: the process issues an I/O read
                %{event completed}: DMA controller interrupts CPU signalling the
                %  completion of an I/O read
                %{exit system}: process terminates or segmentation fault



      \end{itemize}
  \end{multicols}
\end{document}

