\documentclass[fontsize=5pt]{scrartcl}

%
% Original Page by LinuxMercedes
%

\usepackage[
        nohead,
        nofoot,
        left=0.55in,
        right=0.55in,
        top=0.55in,
        bottom=0.55in,
]{geometry}

\usepackage{amsmath,scalefnt}

\usepackage{graphicx}

\renewcommand*{\arraystretch}{.5}

\usepackage{multicol}
\setlength{\columnsep}{5pt}

\usepackage{helvet}
\renewcommand{\familydefault}{\sfdefault}

\pagenumbering{gobble}

\usepackage{enumitem}
\setlist[itemize]{itemsep=-2pt, itemindent=0pt, leftmargin=*}
\setlist[enumerate]{itemsep=-2pt, itemindent=0pt, leftmargin=*}

\usepackage[compact]{titlesec}
\titlespacing{\section}{-1pt}{-1pt}{-1pt}
\titlespacing{\subsection}{-1pt}{-1pt}{-1pt}

\usepackage{listings}

%Y hoy yo reza que no empleo ni alma pobre encontrará esta magica negra.
%This is a custom 'tight' matrix for this cheatsheet. It's ugly.
\newenvironment{tmatrix}%
{ 
  %\scalefont{.5}
  %\setlength{\tabcolsep}{5pt}
  $\left[\hspace{-3.5pt}\begin{array}{c@{\hspace{1pt}}c@{\hspace{1pt}}c@{\hspace{1pt}}|@{\hspace{0pt}}c}
}%
{
   \end{array}\hspace{-3.5pt}\right]$
}

\newenvironment{tmatrix3}%
{ 
  %\scalefont{.5}
  %\setlength{\tabcolsep}{5pt}
  $\left[\hspace{-3.5pt}\begin{array}{c@{\hspace{1pt}}@{\hspace{1pt}}c@{\hspace{1pt}}c@{\hspace{3pt}}}
}%
{
   \end{array}\hspace{-3.5pt}\right]$
}

\newenvironment{tmatrix1}%
{ 
  $\left[\hspace{-3.5pt}\begin{array}{c@{\hspace{3pt}}}
}%
{
   \end{array}\hspace{-3.5pt}\right]$
}


%was 3 3 3 3
\DeclareMathSizes{3pt}{3pt}{3pt}{3pt}

\begin{document}

\begin{multicols}{3}
  \section{Definitions}
    \begin{itemize}
      \item \textbf{Operating Systems}: exploit hardware resources (many processors), provide a set of services to users, manage secondary memory and I/O, %
          and provide networking/comm. support. Elements of an OS:
      \begin{itemize}
        \item Processor
        \item Main Memory (real/primary)
        \item I/O Modules: secondary memory devices, comms equipment, terminals
        \item System bus: communication among processors, memory, and I/O modules
       \end{itemize}
       \item \textbf{Interrupts}: interrupts the normal sequence of execution. Improve processing efficiency and lets the processor not wait on things.
       \item \textbf{ISR}: Interrupt service routine, handles interrupts. Whenever  there is an interrupt, control is transferred to this program.  %
                       It determines the nature of the interrupt and performs the necessary actions to handle it.


   \section{Memory Management}
      \subsection{Requirements}
        \begin{itemize}
          \item \textbf{Relocation}: Programmer/compiler does not know where program will be in memory. Program may also be 
                    moved around during execution. Memory references in the code must be translated.
          \item \textbf{Protection}: Processes shouldn't be allowed to access other's memory without permission. Impossible to know
                    values since program can be relocated, which requires runtime checks. OS cannot anticipate all accesses of a program.
          \item \textbf{Sharing}: Allow several processes to share the same portion of memory. Better to allow each
                    process access to the same copy of a program than have separate copies.
        \end{itemize}
        \subsection{Methods - Fixed Partitioning}
        \begin{itemize}
          \item Memory is divided into partitions which are assigned on demand. These can be equal or variable-sized.
          \item Inefficient, any program, no matter how small, occupies and entire partition.  
          \item \textbf{Strategies}: Equal-sized requires no strategy. Unequal typicall assigns each process to the smallest partition which it can fit,
                wasting the least amount of memory.
          \item \textbf{Partition Assignment}: Can give each partition a queue or have a main queue for the entire memory for processes to wait.
        \end{itemize}
        \subsection{Methods - Dynamic Partitioning}
        \begin{itemize}
          \item Processes are allocated exactly as much memory as required.
          \item Eventually holes start appearing in memory (unallocated bits when processes finish).
          \item \textbf{External Fragmentation}: Chunks of unallocated memory left from processes finishing.
          \item Must use \textbf{compation} to shift processes so they are contiguous and free memory is in one block.
        \end{itemize}
        \subsection{Strategies for Dynamic Partitioning}
        \begin{itemize}
          \item \textbf{First-fit}: fastest, however can have many processes allocated at front end that must be searched over
                while searching for a new block.
          \item \textbf{Best-fit}: Worst performer, chooses block that is closest in size to request. Causes more compation to be
                required as external fragments are the smallest too.
          \item \textbf{Next-fit}: Like first fit, but searches for free blocks beyond the last allocated block. Results in event distribution of free block
                in memory.
        \end{itemize}
        \subsection{Methods - Buddy System}
        \begin{itemize}
          \item Entire space is treated as a single block of $2^{u}$. If a request of size s is such that $2^{u-1}<s\leq2^{u}$ then
          the entire block of $2^{u}$ is allocated.
          \item Otherwise, split into two equal buddies.
          \item Splitting continues until smallest block $\geq$ s is generated.
          % If this is on the test, slide 10 of chap7 has an example that looks decent...
          % http://i.brod.es/dpaEb.png
         \end{itemize}
    \end{itemize}
    
    \section{Virtual Memory}
      % I'm skipping tons of stuff I _know_ here because of homework 2
      \begin{itemize}
       \item \textbf{Page Fault}: Page table indicates that virtual address isn't in memory, os exception handler is invoked to move data from disk to memory.
       \item \textbf{Translation Lookaside Buffer}: High speed cache to look up page table entries. Stores most recently used page table entries. Uses associative mapping (many page
              numbers can map to the same TLB index).
       \begin{itemize}
          \item Given a virtual address, the processor examines the TLB.
          \item If a page table entry is present, it's a ``hit'' and the frame number is returned which leads to the read address.
          \item If it isn't (``miss''), the page number is used to look it up, and the TLB is updated with this data.
       \end{itemize}
       % Optional diagram for how this works Slide 20 Chap8-1
      \end{itemize}
      \subsection{Page Sizes}
      \begin{itemize}
       \item Multi-level page tables allow for a new page table above another, which then maps to other entries. 
       \item Smaller page side leads to less internal fragmentation, larger page tables, and page tables ending up in virtual memory (as they are so large). This can, in turn, lead to double page faults.
       \item Small page sizes means more pages fit in memory, leading to fewer page faults. This is due in part to smaller pages that are used frequently staying in memory, instead of large chunks with a small bit
             used taking up space.
       \item Secondary memory is designed to transfer large chunks of data efficiently, which favors large page sizes.
       \item Larger page sizes leads to some useless references being in memory taking up space.
      \end{itemize}
      % Paging algorithms, including OPT which requires prior knowledge for it to work... keeps pages in memory just right
  \section{Uniprocessor Scheduling}
    \subsection{Aims}
      \begin{itemize}
       \item \textbf{Response Time}: time it takes a system to reactive to a given input (reduce).
       \item \textbf{Turnaround time}: (TAT) total time spent in the system, waiting time + service time.
       \item \textbf{Throughput}: jobs per minute (inverse of TAT), maximize.
      \end{itemize}
      \subsection{Types}
      \begin{itemize}
       \item \textbf{Long-Term Scheduling}
       \begin{itemize}
         \item Determines which programs are admitted to system for processing
         \item Controls degree of multiprogramming
         \item Which job to admit? FCFS, Priority, expected exec time, I/O reqs
       \end{itemize}
       \item \textbf{Medium-Term Scheduling}
       \begin{itemize}
         \item Part of the swapping function
         \item Swapping-in decision is based on the need to manage the degree of multiprogramming
       \end{itemize}
       \item \textbf{Short-term scheduling} is the \textbf{dispatcher}. It executes most frequently and is invoked when events occur. Including clock interrupts, I/O, OS calls, signals, etc.
      \end{itemize}
      \subsection{Factors}
        \begin{itemize}
        \item Priorities: scheduler should choose higher priority processes over lower priority ones. Uses many ready queues to represent each level. Lower priority processes can suffer starvation.
        \item Decision Modes:
          \begin{itemize}
            \item \textbf{Nonpreemptive}: Once a process is in the running state, it will continue until termination or blocks self.
            \item \textbf{Preemptive}: Currently running process can be interrupted and moved to the ready state due to external event. No single process can monopolize processor for long.
          \end{itemize}
        \end{itemize}
      \subsection{Schedulers}
        \begin{itemize}
         \item First-Come-First-Serve (FCFS): nonpreemptive scheduler where oldest process is scheduled to run next.
         \begin{itemize}
           \item Advantage: favors CPU-bound processes. I/O processes wait for CPU bound ones to finish.
           \item Disadvantage: Short processes can wait for a long time before running.
         \end{itemize}
         \item Round Robin (RR): preemptive based on clock (time sliced) interrupt at regular intervals. When an interrupt occurs, process is placed into ready and next job runs.
         \item Shortest Process Next (SPN): Nonpreemptive, process with shortest \textbf{expected} processing time is next. For batch jobs, user is required to estimate the running time
               . For interactive jobs, OS predicts it. Short processes get priority here and long ones can be starved.
         \item Shortest Remaining Time (SRT): Preemptive version of \^. Achieves better turnaround time to \^ as a short job is given immediate preference to a long one. Hard to estimate remaining time.
         \item Highest Response Ratio Next (HRRN): Nonpreemptive, uses R= (time spent waiting + service time ) / service time decide next process. No starvation possible, shorter jobs preferred.
        \end{itemize}
  \section{Multiprocessing}
    \subsection{Issues}
      \begin{itemize}
        \item Multiprogramming usage--- should we allow one application to lock up several cores (maximum speedup)?
        \item Unless a single queue is used for scheduling, it becomes more difficult to maintain specific disciplines.
        \item When a single queue is used, a single process can be schedule to run on any processor (master node needed).
        \item Preemptive schemes (RR) are costly to implement with a single queue approach.
      \end{itemize}
     \subsection{Real-time systems}
       \begin{itemize}
         \item Tasks or processes attempt to control/react to events which must keep up.
         \item Correctness depends on both result AND time at delivery.
         \item Critical that the system is reliable, sometimes with failsafe (traffic lights).
         \item Sometimes include small size, fast context switches, prioritization of scheduling, and special alarms/timeouts.
         \item \textbf{Hard real-time task}: must meet the deadline, causes catastrophic failure/errors w/o.
         \item \textbf{Soft real-time task}: has a deadline that is desired. Makes sense to complete even if late.
         \item \textbf{Periodic tasks}: Only one unit per period T, exactly T apart.
         \item \textbf{Aperiodic task}: has a deadline by which it must finish/start/both.
       \end{itemize}
      \subsection{Deadline scheduling}
        \begin{itemize}
          \item Real time application, not concerned with fairness/response time but with prioritizing tasks based on deadlines.
          \item Earliest deadlines typically scheduled first.
          % there's 2 example slides here, slide 13/14 on chap10.
          \item Best to know each first:
          \begin{itemize}
            \item Ready time (periodic tasks know this)
            \item Starting/completion deadline
            \item Processing time and resource reqs.
            \item Priority
           \end{itemize}
        \end{itemize}
      \subsection{Rate Monotonic Scheduling}
        \begin{itemize}
          \item Assigns priorities to tasks on basis of periods.
          \item Highest priority tasks have the shortest period.
          \item Always works if the below is satisfied:\\
          $\frac{C_1}{T_1} + \frac{C_2}{T_2} + \dots + \frac{C_n}{T_n} \le n(2^{\frac{1}{n}}-1)$
          \item Has industrial applications.
          \item Higher priority tasks have higher frequencies (hz)
          \item n is the number of tasks, C is cycle time T is processing time?
        \end{itemize}
        % Unix SVR scheduling skipped
  \section{next}



      





    
  \section{Glossary}
    \begin{itemize}
      \item \textbf{Address Space} – The range of addresses available to a computer program
      \item \textbf{Address Translator} – A functional unit that transforms virtual addresses to real addresses
      \item \textbf{Asynchronous Operation} – An operation that occurs without a regular or predictable time relationship to a specified event; 
                              for example, the calling of an error diagnostic routine that may receive control at any time during the execution of a program.
      \item \textbf{Atomic Operation} – An operation which cannot be split into more operations. Examples: primitive types: read and write, fetch-and-add, compare-and-swap. 
      \item \textbf{Busy waiting} – the repeated execution of a loop of code while waiting for an event to occur
      \item \textbf{Consumable Resource} – only available once such as message send, also can cause \textbf{deadlocks}.
      \item \textbf{Context Switch} – an operation that switches the processor from one process to another, by saving all the process control block, registers, and other information for the first and 
                                      replacing them with the process information for the second
      \item \textbf{Direct Memory Access (DMA)} – a form of I/O in which a special module, called a DMA module, controls the exchange of data between main memory and an I/O device.  
                                  The processor sends a request for the transfer block of data to the DMA module and is interrupted only after the entire block has been transferred.
      \item \textbf{File Allocation Table (FAT)} – a table that indicates the physical location on secondary storage of the space allocated to a file.  There is one file allocation table for each file.
      \item \textbf{Job Control language (JCL)} – a problem-oriented language that is designed to express statements in a job that are used to identify the job or to describe its requirements to an operating system
      \item \textbf{Kernel} – a portion of the operating system that includes the most heavily used portions of software.  Generally, the kernel is maintained permanently in main memory.  
              The kernel runs in a privileged mode and responds to calls from processes and interrupts from devices.
      \item \textbf{Message} – a block of information that may be exchanged between processes as a means of communication.
      \item \textbf{Preemption} – reclaiming a resource from a process before the process has finished using it.
      \item \textbf{Process Image} – all of the ingredients of a process, including program, data, stack, and process control block
      \item \textbf{Reusable Resource} – can be used several times, cause for \textbf{deadlock} often. semaphore, CPU, etc
      \item \textbf{Race Condition} – an undesirable situation that occurs when a device or system attempts to perform two or more operations at the same time, but because of the nature of the device or system, 
      the operations must be done in the proper sequence in order to be done correctly.
      \item \textbf{Round Robin} – a scheduling algorithm in which processes are activated in a fixed cyclic order.  Those which cannot proceed because they are waiting for some event 
                    (e.g., termination of a child process or an input/output operation) simply return control to the scheduler.
      \item \textbf{Time Sharing} – the concurrent use of a device by a number of users
      \item \textbf{Time Slicing} – a mode of operation in which two or more processes are assigned quanta of time on the same processor
    \end{itemize}
    
    
  \section{QA Pool}
    \subsection{T/F - Process Dynamics}

    \subsection{OS Related Questions}


  \end{multicols}
\end{document}

